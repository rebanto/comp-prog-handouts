\documentclass[12pt]{article}
\usepackage{amsmath, amssymb}
\usepackage{geometry}
\usepackage{enumitem}
\usepackage{listings}
\geometry{letterpaper, margin=1in}

\usepackage{minted}

\setlength{\parindent}{0pt}
\setlength{\parskip}{0.5em}

% Listings setup
\lstset{
    basicstyle=\ttfamily\small,
    frame=single,
    numbers=none,
    tabsize=4,
    showstringspaces=false,
    breaklines=true
}

\begin{document}

\begin{center}
    {\small \textbf{IA Competitive Programming Club}} \\[6pt]
    {\Large Time Complexity Worksheet} \\[12pt]
    {\small September 23, 2025}
\end{center}
\vspace{3em}


\begin{enumerate}[label=\textbf{\arabic*.}]
    \item You are asked to sum the elements of an array of size $n$ by scanning it once.  
    What is the time complexity?  
    \vspace{2cm}

    \item You are given the following code:
        
        \begin{minted}[frame=single,framesep=10pt]{python}
for i in range(1, n+1):
    for j in range(1, n+1):
        print(i, j)
        \end{minted}
        
    How many times does \texttt{print} execute? State the time complexity.  
    \vspace{2.5cm}

    \item A programmer writes:
    \begin{minted}[frame=single,framesep=10pt]{python}
i = 1
while i <= n:
    i = i * 3
    \end{minted}
    How many times does the loop run? State the time complexity.  
    \vspace{2.5cm}

    \item You are asked to check if a number $n$ is prime by dividing it by all integers up to $\sqrt{n}$.  
    What is the time complexity? 
    \vspace{2cm}

    \item Consider the following:
    \begin{minted}[frame=single,framesep=10pt]{python}
for i in range(1, n+1):
    j = i
    while j > 0:
        j = j // 2
    \end{minted}
    Estimate the time complexity of this program.
    \vspace{2.5cm}

    \item A system processes $m$ queries, and each query takes $O(\log n)$ time.  
    What is the total time complexity in terms of $m$ and $n$?  
    \vspace{2cm}

    \item Analyze the following:
    \begin{minted}[frame=single,framesep=10pt]{python}
for i in range(1, n+1):
    for j in range(1, i+1):
        print(i, j)
    \end{minted}
    How many times does \texttt{print} run? State the time complexity.  
    \vspace{8.5cm}

    \item Consider this loop:
    \begin{minted}[frame=single,framesep=10pt]{python}
i = n
while i > 0:
    i = i // 2
    \end{minted}
    How many iterations occur? Give the time complexity.  
    \vspace{2.5cm}

    \item A recursive function is written as:
    \begin{minted}[frame=single,framesep=10pt]{python}
def f(n):
    if n == 1:
        return
    f(n-1)
    f(n-1)
    \end{minted}
    Solve for its time complexity.  
    \vspace{2.5cm}

    \item Finally, consider:
    \begin{minted}[frame=single,framesep=10pt]{python}
for i in range(1, n+1):
    for j in range(1, int(sqrt(i))+1):
        print(i, j)
    \end{minted}
    How many times does \texttt{print} run? Give the overall time complexity.  
    \vspace{4.5cm}

        \item Farmer John lines up his $n$ cows, each with an ID number.  
    He wants to check every possible pair of cows $(i,j)$ with $i < j$, 
    and for each pair, he compares their IDs digit by digit. 
    Each ID has at most $\log n$ digits.  

    Write the time complexity of his process in terms of $n$.  
    \vspace{3cm}

    % \item Farmer John notices his cows sometimes form groups.  
    % For every integer $k$ from $1$ to $n$, he checks all multiples of $k$ up to $n$.  
    % (That is, for each $k$, he runs a loop over $k, 2k, 3k, \dots \leq n$.)  

    % What is the total running time across all values of $k$?

    \item Farmer John lines up $n$ cows, each labeled with a unique number from $1$ to $n$. He notices that his cows can be arranged into groups: for each integer $k$ from $1$ to $n$, there is a group consisting of all cows whose labels are multiples of $k$.
    \\

    Farmer John carefully lists every cow in every group, going group by group. A cow may appear in multiple groups, and each appearance is counted separately.
    \\
    
    How many total listings does Farmer John make across all groups?
    \vspace{3cm}

    \item Farmer John defines a recursive function on his farm:  
    \begin{minted}[frame=single,framesep=10pt]{python}
def FJ(n):
    if n <= 1:
        return
    for i in range(1, n+1):
        pass   # constant work
    FJ(n//2)
    FJ(n//2)
    \end{minted}
    Solve for the time complexity.  
    \vspace{3cm}

    % Sreyas hypothesis: n (log n)^2

\end{enumerate}

\end{document}
