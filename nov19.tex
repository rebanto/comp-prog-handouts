\documentclass[11pt]{article}
\usepackage[margin=0.7in]{geometry}
\usepackage{amsmath, amssymb}
\usepackage{enumitem}
\usepackage{xcolor} % Useful for setting colors, though not used here, often good practice
\setlength{\parskip}{6pt}
\setlength{\parindent}{0pt}

\begin{document}

\begin{center}
    {\small \textbf{IA Competitive Programming Club}} \\[6pt]
    {\Large Sets \& Maps Worksheet} \\[12pt]
    {\small November 19, 2025}
\end{center}

\vspace{0.2em}

%%%%%%%%%%%%%%%%%%%%%%%%%%%%%%%%
%%%% SECTION 1
%%%%%%%%%%%%%%%%%%%%%%%%%%%%%%%%

\section*{1. Quick Checks}

\begin{enumerate}[label=\textbf{1.\arabic*}.]

\item Fill in the expected time (average case):
$$
\text{set.add: } \rule{2.5cm}{0.4pt}, \quad
\text{(x in s): } \rule{2.5cm}{0.4pt}, \quad
\text{dict lookup: } \rule{2.5cm}{0.4pt}.
$$

\item True or False: A set removes duplicates automatically. \quad
\textbf{Circle: True \quad False}

\item Circle all types that can be dictionary keys: \quad
\textbf{int \quad str \quad tuple \quad list}

\item Why can’t you index a Python list using values up to $10^{12}$?
\vspace{0.1cm}
\vspace{0.3cm} % Add some space after the line for better look

\end{enumerate}

\vspace{0.1em}

%%%%%%%%%%%%%%%%%%%%%%%%%%%%%%%%
%%%% SECTION 2
%%%%%%%%%%%%%%%%%%%%%%%%%%%%%%%%

\section*{2. Dry-Run Problems}

\begin{enumerate}[label=\textbf{2.\arabic*}.]

\item Predict the output:
\begin{verbatim}
s = set()
for x in [10, 10, 3, 5, 3]:
    s.add(x)
print(len(s))
\end{verbatim}
Output: \rule{3cm}{0.4pt}

\vspace{0.2cm}

\item Predict the output:
\begin{verbatim}
m = {}
m["a"] = 4
m["b"] = 7
m["a"] += 5
print(m["a"], m.get("c", 0))
\end{verbatim}
Output: \rule{4cm}{0.4pt}

\vspace{0.2cm}

\item Predict the output:
\begin{verbatim}
nums = [5,1,5,8,8,8]
freq = {}
for x in nums:
    freq[x] = freq.get(x, 0) + 1
print(freq[5] + freq[8])
\end{verbatim}
Output: \rule{3cm}{0.4pt}

\vspace{0.2cm}

\item Predict the output:
\begin{verbatim}
A = set([2,4,6])
B = set([4,2,6,6])
print(A == B)
\end{verbatim}
Output: \rule{2cm}{0.4pt}

\end{enumerate}

\vspace{0.5em}

%%%%%%%%%%%%%%%%%%%%%%%%%%%%%%%%
%%%% SECTION 3
%%%%%%%%%%%%%%%%%%%%%%%%%%%%%%%%

\section*{3. Applied and Numerical Problems}

\begin{enumerate}[label=\textbf{3.\arabic*}.]

\item You insert $400{,}000$ distinct numbers into a set.
There are $10{,}000$ buckets.
What is the number of expected items per bucket? 

\vspace{0.5cm}

\item A dictionary \texttt{freq} stores word counts.
If the most common word appears $6000$ times and the least appears once, what does:
\begin{verbatim}
max(freq.values()) - min(freq.values())
\end{verbatim}
produce?
Answer: \rule{3cm}{0.4pt}

\vspace{0.5cm}

\item Write a one-line Python expression that prints the smallest value in a dictionary \texttt{price}:

\vspace{0.5cm}

\item A program performs $N$ membership checks on a set of size $N$.
Expected total time (in $O$-notation):

\vspace{0.5cm}

\item Complete dictionary assignment and lookup:

\begin{verbatim}
m = {}      # keys up to 10^12

def store(k, v):
    _________________________

def retrieve(k):
    return m.get(k, ____)
\end{verbatim}

\vspace{0.5cm}

\item Only $80{,}000$ integer keys (up to $10^{12}$) will be stored.
Circle the best data structure:
\\[6pt]
\textbf{dict \quad list \quad array \quad set}

Write one short reason:
\vspace{0.1cm}

\end{enumerate}


\end{document}